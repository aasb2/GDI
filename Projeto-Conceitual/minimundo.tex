\documentclass[12pt, a4paper]{article}
%\usepackage[latin1]{inputenc}
\usepackage[utf8]{inputenc}
\usepackage[brazil]{babel}
\usepackage{indentfirst}
\usepackage{setspace}
\begin{document}

\begin{titlepage}
\begin{center}
{\large Universidade Federal de Pernambuco}\\[0.2cm]
{\large Centro de Informática}\\[0.2cm]
{\large Gerenciamento de Dados e Informação}\\[3.7cm]
{\bf \huge MINIMUNDO}\\[5.cm]
\end{center}
{\large {\bf Equipe:}\\[0.2cm]
Antonio Carlos Gomes Marinho (acgm)\\[0.2cm]
Arthur Abrahao Santos Barbosa (aasb2)\\[0.2cm]
Jhonatan Kennedy Pires de Andrade (jkpa)\\[0.2cm]
Joao Pedro Ribeiro da Silva Dias (jprsd)\\[0.2cm]
Marcos Antonio Vital de Lima (mavl)\\[0.2cm]
Rafael Henrique Ayres (jkpa)\\[0.2cm]}
{\large {\bf Professor:} Robson do Nascimento Fidalgo}\\[2cm]
\begin{center}
{\large Recife}\\[0.2cm]
{\large 2021}\\[0.2cm]
\end{center}
\end{titlepage}
\onehalfspacing
\section{Descrição do Minimundo}

Um \textbf{funcionário} tem \underline{CPF}, nome, salário, data de nascimento, um ou mais telefones e um endereço (formado por descrição e CEP). Um \textbf{funcionário} pode supervisionar outros \textbf{funcionários}, os quais podem ser supervisionados por um único funcionário.

Ao longo de sua vida na empresa, o \textbf{funcionário} pode trabalhar em vários \textbf{departamentos}, 
nos quais podem trabalhar vários \textbf{funcionários}. Além do histórico dos \textbf{departamentos} em que cada 
\textbf{funcionário} trabalhou, pede-se que seja guardado o \underline{código}, nome e a descrição dos \textbf{departamentos}. 
Um \textbf{funcionário} pode chefiar um \textbf{departamento}, o qual deve ser chefiado por um único \textbf{funcionário.}

Uma \textbf{tarefa} só é definida se fizer parte de um \textbf{projeto de jogos}, sendo identificada através do 
\underline{código} do \textbf{projeto de jogo} ao qual ela pertence e o seu ID. Possui como atributos ID e descrição. 
Uma \textbf{tarefa} pode ser executada por vários \textbf{funcionários}, e um \textbf{funcionário} pode executar várias \textbf{tarefas}.
 Na execução de uma tarefa o(s) \textbf{funcionário(s)} pode(m) ou não utilizar \textbf{Tecnologias Licenciadas}. 
 Uma \textbf{tecnologia licenciada} possui \underline{marca registrada}, descrição e nome da empresa cedente. 

Todo \textbf{projeto de jogos} tem um \underline{código}, nome, custo, status, data de início e prazo. Há duas subclasses de projeto: \textbf{mobile}, cuja plataforma deve ser especificada; e \textbf{PC}, que possuem categoria e classificação etária. Um \textbf{funcionário} pode gerenciar vários \textbf{projetos de jogos}, que devem ser gerenciados por um único \textbf{funcionário}.

Um \textbf{distribuidor} tem \underline{CNPJ}, nome e e-mail. Ele pode distribuir vários \textbf{projetos de jogos}, que podem ser distribuídos por vários \textbf{distribuidores}. Cada negociação de um jogo com um \textbf{distribuidor} será feita por um \textbf{funcionário} e possuirá um número de contrato, e será identificada através da data em que ocorreu. 


\end{document}